\documentclass[18pt, a4paper]{article}


\usepackage{geometry}
 \geometry{
 a4paper,
 total={170mm,257mm},
 left=20mm,
 top=10mm,
 }
\usepackage{graphicx}


	\begin{document}
		
				
		\title{\textbf{ PARKING SPACES AROUND MAKERERE UNIVERSITY}\\ }

		
		\author{MAJANGA JOSEPH}

		\date {\today}

		\maketitle

			
                         \section{Contents}
                  Acknowledgements..................................................2\\
                  Acronyms ..............................................................3\\
                  Executive Summary ................................................4\\
                           4.1............background\\
                           4.2............methodology\\
                           4.3............key findings\\
                  Introduction...........................................................5\\
                           5.1.............objectives\\
                  Methodology...........................................................6\\

                     \section{Acknowledgements}
I wish to thank Makerere University for allowing me access and to document their PARKING LOT INFORMATION.  In particular, I wish to thank Professor Engineer Bainomugisha.  the Head of Department Computer Science, for his keen interest and guidance in the development of this study.\\


I also wish to thank my supervisor, Professor Earnest Mwebaze for the academic support and assistance throughout the writing of the report.\\



			\section{Acronyms}
XML		Extensible Mark-up Language \\
ODK		Open Data Kit\\			


			\section {Executive Summary}
				\subsection{Background}
The system mainly collects data from colleges, schools AND halls of residence and with all these places, the parking space are enough. i.e if for example a student drives he OR she can park at a hall of residence and when they go for lectures, they can park at the college parking areas.

\subsection{Methodology}

The data in this research was collected using electronic methods. An XML (Extensible Mark-up Language ) was created and uploaded on the (aggregate) server, the link to the aggregate entered in an android application(ODK collect) to link up the smart phone application to the server.

				\subsection{Key Findings}
Some people really do not where the most parking lots so they end up parking in very crowded areas which may lead to increased insecurity of the cars and the property inside the cars.\\

•	Some parking lots are not safe for the night because they lack security or security guards hence cars can be vandalised.\\



				\section{Introduction}

Makerere University is a big place indeed thus the need to collect data of ample parking spaces where staff and students can park vehicles around the University. And since it is big, the people need to know the various areas where to park because they can’t all park in one place.\\

				\subsection{Objectives}




•	To help people find parking areas (spaces) around Makerere University.\\
•	To find out the maximum amount of cars can take when full\\ 
•	To find out the location of the parking area either college or school or hall of residence:\\


				\section{Methodology}


                                              \subsection{Instruments}

\textbf{Aggregate server}: this was used a platform to centrally manage the data, and every data that was collected was sent to this server and can later be used for visualization.\\


\textbf{ODK collect}: A mobile application that was used to enter the data collected including geo position of each entry and a picture of the parking lots from which data was obtained.\\



\textbf{XML FORM BUILDER}: this is an online plat form that created the form named wifi form and giving data values to each entry.\\

\includegraphics{f}

			
			\subsection{Data Collection}

     
         The data was collected  a by phone where the researcher (me) walked around makerere university taking pictures with my phone with the help of the odk form.\\

	Geo position was obtained using the satellite readings on the application while running on the smart phone.\\
   \includegraphics{4}

	The maximum number of cars each parking lot can take i.e below is the COCIS parking lot.(the max number of cars of each parking lot is found in the form)\\
   
                         \includegraphics{5}

                       \includegraphics{6}

			\subsection{Data Analysis}

After collecting the data using the ODK collection tool, an internet connection would later be used to upload the obtained entries directly on the aggregate server which is on the site http://project-167919.appspot.com. On the same site, the data visualization of bar graphs, pie charts and map view of the coordinates \\

			
				\section{Results}

The results of the research can be found at the aggregate server where they can be uploaded.\\

			

		\section{References}
          The FEMA NIGHT GUARD(names witheld) who gave me some information about the FEMA parking lot.



			

				
	

	\end{document}
